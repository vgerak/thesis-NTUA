% !TEX root = ../thesis.tex

\chapter{Εισαγωγή}
% Αυτό το κεφάλαιο είναι το ελληνικό υποκατάστατο του πλήρους κειμένου;
% Αν ναι, θα πρότεινα να ονομαστεί "Περίληψη", από το [extended] Abstract.
% Τα υπο-κεφάλαιά του θα είναι 1-1 τα κεφάλαια του πρωτότυπου αγγλικού
% κειμένου. Θα πρότεινα να μην μας απασχολήσει τώρα. Θα το γράψεις προς το
% τέλος, όταν ήδη το κείμενο έχει αποκτήσει μορφή.
% Έχω κάνει κάτι ανάλογο κι εδώ, αυτό είδα και προτείνω να ακολουθήσουμε
% τώρα:
% http://www.cslab.ece.ntua.gr/~vkoukis/files/vkoukis-phd.pdf

\section{Κίνητρο}
  Μπλα μπλα μπλα, μπλα μπλα, μπλα μπλα μπλα, μπλα μπλα μπλα μπλα,
  μπλα μπλα μπλα, μπλα μπλα μπλα, μπλα, μπλα μπλα μπλα, μπλα μπλα,

\section{Κύρια σημεία της εργασίας}
  Μπλα μπλα μπλα, μπλα μπλα, μπλα μπλα, μπλα, μπλα μπλα
  μπλα, μπλα μπλα, μπλα, μπλα, μπλα μπλα, μπλα

\section{Οργάνωση κειμένου}
  Μπλα μπλα μπλα, μπλα μπλα, μπλα μπλα μπλα, μπλα μπλα μπλα μπλα,
  μπλα μπλα μπλα, μπλα μπλα μπλα, μπλα, μπλα μπλα μπλα, μπλα μπλα,

\section{Συνοπτική παρουσίαση της εφαρμογής}
  Μπλα μπλα μπλα, μπλα μπλα, μπλα μπλα μπλα, μπλα μπλα μπλα μπλα,

\section{Συνοπτική παρουσίαση των πειραματικών αποτελεσμάτων}
  Μπλα μπλα μπλα, μπλα μπλα, μπλα μπλα μπλα, μπλα μπλα μπλα μπλα

\setcounter{chapter}{0}

\chapter{Introduction}

\section{Motivation}
  Lorem ipsum dolor sit amet, consectetur adipiscing elit. Donec euismod ante
  non felis condimentum efficitur. Nunc vel pretium diam.

\section{Thesis contribution}
  Lorem ipsum dolor sit amet, consectetur adipiscing elit. Donec euismod ante
  non felis condimentum efficitur. Nunc vel pretium diam.

\section{Chapter outline}
  Lorem ipsum dolor sit amet, consectetur adipiscing elit. Donec euismod ante
  non felis condimentum efficitur. Nunc vel pretium diam.

\section{Brief description of the application}
  Lorem ipsum dolor sit amet, consectetur adipiscing elit. Donec euismod ante
  non felis condimentum efficitur. Nunc vel pretium diam.
